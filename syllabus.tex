% Options for packages loaded elsewhere
\PassOptionsToPackage{unicode}{hyperref}
\PassOptionsToPackage{hyphens}{url}
\PassOptionsToPackage{dvipsnames,svgnames,x11names}{xcolor}
%
\documentclass[
]{scrartcl}

\usepackage{amsmath,amssymb}
\usepackage{iftex}
\ifPDFTeX
  \usepackage[T1]{fontenc}
  \usepackage[utf8]{inputenc}
  \usepackage{textcomp} % provide euro and other symbols
\else % if luatex or xetex
  \usepackage{unicode-math}
  \defaultfontfeatures{Scale=MatchLowercase}
  \defaultfontfeatures[\rmfamily]{Ligatures=TeX,Scale=1}
\fi
\usepackage{lmodern}
\ifPDFTeX\else  
    % xetex/luatex font selection
\fi
% Use upquote if available, for straight quotes in verbatim environments
\IfFileExists{upquote.sty}{\usepackage{upquote}}{}
\IfFileExists{microtype.sty}{% use microtype if available
  \usepackage[]{microtype}
  \UseMicrotypeSet[protrusion]{basicmath} % disable protrusion for tt fonts
}{}
\makeatletter
\@ifundefined{KOMAClassName}{% if non-KOMA class
  \IfFileExists{parskip.sty}{%
    \usepackage{parskip}
  }{% else
    \setlength{\parindent}{0pt}
    \setlength{\parskip}{6pt plus 2pt minus 1pt}}
}{% if KOMA class
  \KOMAoptions{parskip=half}}
\makeatother
\usepackage{xcolor}
\setlength{\emergencystretch}{3em} % prevent overfull lines
\setcounter{secnumdepth}{-\maxdimen} % remove section numbering
% Make \paragraph and \subparagraph free-standing
\makeatletter
\ifx\paragraph\undefined\else
  \let\oldparagraph\paragraph
  \renewcommand{\paragraph}{
    \@ifstar
      \xxxParagraphStar
      \xxxParagraphNoStar
  }
  \newcommand{\xxxParagraphStar}[1]{\oldparagraph*{#1}\mbox{}}
  \newcommand{\xxxParagraphNoStar}[1]{\oldparagraph{#1}\mbox{}}
\fi
\ifx\subparagraph\undefined\else
  \let\oldsubparagraph\subparagraph
  \renewcommand{\subparagraph}{
    \@ifstar
      \xxxSubParagraphStar
      \xxxSubParagraphNoStar
  }
  \newcommand{\xxxSubParagraphStar}[1]{\oldsubparagraph*{#1}\mbox{}}
  \newcommand{\xxxSubParagraphNoStar}[1]{\oldsubparagraph{#1}\mbox{}}
\fi
\makeatother


\providecommand{\tightlist}{%
  \setlength{\itemsep}{0pt}\setlength{\parskip}{0pt}}\usepackage{longtable,booktabs,array}
\usepackage{calc} % for calculating minipage widths
% Correct order of tables after \paragraph or \subparagraph
\usepackage{etoolbox}
\makeatletter
\patchcmd\longtable{\par}{\if@noskipsec\mbox{}\fi\par}{}{}
\makeatother
% Allow footnotes in longtable head/foot
\IfFileExists{footnotehyper.sty}{\usepackage{footnotehyper}}{\usepackage{footnote}}
\makesavenoteenv{longtable}
\usepackage{graphicx}
\makeatletter
\newsavebox\pandoc@box
\newcommand*\pandocbounded[1]{% scales image to fit in text height/width
  \sbox\pandoc@box{#1}%
  \Gscale@div\@tempa{\textheight}{\dimexpr\ht\pandoc@box+\dp\pandoc@box\relax}%
  \Gscale@div\@tempb{\linewidth}{\wd\pandoc@box}%
  \ifdim\@tempb\p@<\@tempa\p@\let\@tempa\@tempb\fi% select the smaller of both
  \ifdim\@tempa\p@<\p@\scalebox{\@tempa}{\usebox\pandoc@box}%
  \else\usebox{\pandoc@box}%
  \fi%
}
% Set default figure placement to htbp
\def\fps@figure{htbp}
\makeatother
% definitions for citeproc citations
\NewDocumentCommand\citeproctext{}{}
\NewDocumentCommand\citeproc{mm}{%
  \begingroup\def\citeproctext{#2}\cite{#1}\endgroup}
\makeatletter
 % allow citations to break across lines
 \let\@cite@ofmt\@firstofone
 % avoid brackets around text for \cite:
 \def\@biblabel#1{}
 \def\@cite#1#2{{#1\if@tempswa , #2\fi}}
\makeatother
\newlength{\cslhangindent}
\setlength{\cslhangindent}{1.5em}
\newlength{\csllabelwidth}
\setlength{\csllabelwidth}{3em}
\newenvironment{CSLReferences}[2] % #1 hanging-indent, #2 entry-spacing
 {\begin{list}{}{%
  \setlength{\itemindent}{0pt}
  \setlength{\leftmargin}{0pt}
  \setlength{\parsep}{0pt}
  % turn on hanging indent if param 1 is 1
  \ifodd #1
   \setlength{\leftmargin}{\cslhangindent}
   \setlength{\itemindent}{-1\cslhangindent}
  \fi
  % set entry spacing
  \setlength{\itemsep}{#2\baselineskip}}}
 {\end{list}}
\usepackage{calc}
\newcommand{\CSLBlock}[1]{\hfill\break\parbox[t]{\linewidth}{\strut\ignorespaces#1\strut}}
\newcommand{\CSLLeftMargin}[1]{\parbox[t]{\csllabelwidth}{\strut#1\strut}}
\newcommand{\CSLRightInline}[1]{\parbox[t]{\linewidth - \csllabelwidth}{\strut#1\strut}}
\newcommand{\CSLIndent}[1]{\hspace{\cslhangindent}#1}

\makeatletter
\@ifpackageloaded{caption}{}{\usepackage{caption}}
\AtBeginDocument{%
\ifdefined\contentsname
  \renewcommand*\contentsname{Table of contents}
\else
  \newcommand\contentsname{Table of contents}
\fi
\ifdefined\listfigurename
  \renewcommand*\listfigurename{List of Figures}
\else
  \newcommand\listfigurename{List of Figures}
\fi
\ifdefined\listtablename
  \renewcommand*\listtablename{List of Tables}
\else
  \newcommand\listtablename{List of Tables}
\fi
\ifdefined\figurename
  \renewcommand*\figurename{Figure}
\else
  \newcommand\figurename{Figure}
\fi
\ifdefined\tablename
  \renewcommand*\tablename{Table}
\else
  \newcommand\tablename{Table}
\fi
}
\@ifpackageloaded{float}{}{\usepackage{float}}
\floatstyle{ruled}
\@ifundefined{c@chapter}{\newfloat{codelisting}{h}{lop}}{\newfloat{codelisting}{h}{lop}[chapter]}
\floatname{codelisting}{Listing}
\newcommand*\listoflistings{\listof{codelisting}{List of Listings}}
\makeatother
\makeatletter
\makeatother
\makeatletter
\@ifpackageloaded{caption}{}{\usepackage{caption}}
\@ifpackageloaded{subcaption}{}{\usepackage{subcaption}}
\makeatother

\usepackage{bookmark}

\IfFileExists{xurl.sty}{\usepackage{xurl}}{} % add URL line breaks if available
\urlstyle{same} % disable monospaced font for URLs
\hypersetup{
  pdftitle={DSAN 5650 Syllabus - Summer 2025},
  pdfauthor={Prof.~Jeff Jacobs},
  colorlinks=true,
  linkcolor={blue},
  filecolor={Maroon},
  citecolor={Blue},
  urlcolor={Blue},
  pdfcreator={LaTeX via pandoc}}


\title{DSAN 5650}
\usepackage{etoolbox}
\makeatletter
\providecommand{\subtitle}[1]{% add subtitle to \maketitle
  \apptocmd{\@title}{\par {\large #1 \par}}{}{}
}
\makeatother
\subtitle{Causal Inference for Computational Social
Science \\ \textit{Wednesdays 6:30-9pm, Online}}
\author{Prof.~Jeff
Jacobs \\[-0.2em] \normalsize{\href{mailto:jj1088@georgetown.edu}{\texttt{jj1088@georgetown.edu}}}}
\date{\normalsize{Summer 2025, Georgetown University}}
% 
\begin{document}
\maketitle


\textbf{Welcome to DSAN 5650: Causal Inference for Computational Social
Science at Georgetown University!}

The course meets on \textbf{Wednesdays from 6:30-9pm online via
\href{https://maps.app.goo.gl/fVVfDFpp4MEuukXa9}{Zoom}}

\subsection{Course Staff}\label{course-staff}

\begin{itemize}
\tightlist
\item
  Prof.~Jeff Jacobs,
  \href{mailto:jj1088@georgetown.edu}{\texttt{jj1088@georgetown.edu}}

  \begin{itemize}
  \tightlist
  \item
    Office hours (Click to schedule):
    \href{https://jjacobs.me/meet}{Tuesdays, 3:30-6:30pm}
  \end{itemize}
\item
  TA Courtney Green,
  \href{mailto:crg123@georgetown.edu}{\texttt{crg123@georgetown.edu}}

  \begin{itemize}
  \tightlist
  \item
    Office hours by appointment
  \end{itemize}
\item
  TA Wendy Hu,
  \href{mailto:lh1078@georgetown.edu}{\texttt{lh1078@georgetown.edu}}

  \begin{itemize}
  \tightlist
  \item
    Office hours by appointment
  \end{itemize}
\end{itemize}

\subsection{Course Description}\label{course-description}

This course provides students with the opportunity to take the
analytical skills, machine learning algorithms, and statistical methods
learned throughout their first year in the program and explore how they
can be employed towards carrying out rigorous, original research in the
behavioral and social sciences. With a particular emphasis on tackling
the additional challenges which arise when moving from associational to
causal inference, particularly when only observational (as opposed to
experimental) data is available, students will become proficient in
cutting-edge causal Machine Learning techniques such as propensity score
matching, synthetic controls, causal program evaluation, inverse social
welfare function estimation from panel data, and Double-Debiased Machine
Learning.

In-class examples will cover continuous, discrete-choice, and textual
data from a wide swath of social and behavioral sciences: economics,
political science, sociology, anthropology, quantitative history, and
digital humanities. After gaining experience through in-class labs and
homework assignments focused on reproducing key findings from recent
journal articles in each of these disciplines, students will spend the
final weeks of the course on a final project demonstrating their ability
to develop, evaluate, and test the robustness of a causal hypothesis.

\emph{Prerequisites: DSAN 5000, DSAN 5100 (DSAN 5300 recommended but not
required)}

\subsection{Course Overview}\label{course-overview}

The fundamental building block for the course is the idea of a
\textbf{Data-Generating Process (DGP)}. You may have encountered this
concept in passing during other DSAN courses (for example, in DSAN 5100,
a phrase like ``Assume \(X\) is drawn i.i.d. from a Normal distribution
with mean \(\mu\) and variance \(\sigma^2\)'' is a statement
characterizing the DGP of a Random Variable \(X\)), but in this course
we will ``zoom in'' on this concept rather than treating it like a black
box or a footnote to e.g.~a theorem like the Law of Large Numbers.

This deep dive into DGPs is necessary for us here, since our goal in the
course is \textbf{to move from \emph{associational} statements} like
``an increase of \(X\) by one unit is associated with an increase of
\(Y\) by \(\beta\) units'' to \textbf{\emph{causal} statements} like
``increasing \(X\) by one unit \emph{causes} \(Y\) to increase by
\(\beta\) units''. As you'll see in Week 1, the tools from probability
theory and statistics that you learned in DSAN 5100---Random Variables,
Cumulative Distribution Functions, Conditional Probability, and so
on---are necessary but \textbf{not \emph{sufficient}} to analyze data
from a causal perspective.

For example, if we use our tools from DSAN 5000 and DSAN 5100 on some
dataset to discover that:

\begin{itemize}
\tightlist
\item
  The probability that some event \(E_1\) occurs is \(\Pr(E_1) = 0.5\),
  and
\item
  The probability that \(E_1\) occurs \textbf{conditional on} another
  event \(E_0\) occurring is \(\Pr(E_1 \mid E_0) = 0.75\),
\end{itemize}

we unfortunately cannot infer from these two pieces of information that
the occurrence of \(E_0\) \textbf{causes} an increase in the likelihood
of \(E_1\) occurring.

This issue (that \textbf{conditional probabilities} could not be
interpreted causally) at first represented a kind of dead end for
scientists interested in employing probability theory to study causal
relationships\ldots{} In recent decades, however, researchers have built
up what amounts to an additional ``layer'' of modeling tools which
augment the existing machinery of probability theory to address
causality head-on!\footnote{Pearl (2000) represents a key work in this
  field of research, as it essentially brought together different pieces
  of causal models into one unified, rigorous framework.}

For instance, a modeling approach called
\textbf{``\(\textsf{do}\)-Calculus''}, that we will learn in this class,
\emph{extends} the core operations and definitions of probability theory
to allow such an move to deriving causality! It does this by introducing
a \(\textsf{do}(\cdot)\) operator that can be applied to Random
Variables like \(X\), with e.g.~\(\textsf{do}(X = 5)\) representing the
event wherein someone has \textbf{intervened} in a
\textbf{Data-Generating Process} to \textbf{force} the value of \(X\) to
be 5.

With this operator in hand (that is, used alongside an explicit model of
a DGP satisfying a set of underlying axioms which are slightly more
strict than the axioms of probability theory), it turns out that we
\emph{can} make causal inferences using a very similar pair of facts! If
we know that:

\begin{itemize}
\tightlist
\item
  The probability that some event \(E_1\) occurs is \(\Pr(E_1) = 0.5\),
  and
\item
  The probability that \(E_1\) occurs \textbf{conditional on} the event
  \(\textsf{do}(E_0)\) occurring is
  \(\Pr(E_1 \mid \textsf{do}(E_0)) = 0.75\),
\end{itemize}

\textbf{now} we can actually draw the inference that the occurrence of
\(E_0\) \textbf{caused} an increase in the likelihood of \(E_1\)
occurring!

This stylized comparison (between what's possible using ``core''
probability theory and what's possible when we augment it with
additional causal modeling tools) serves as our basic motivation for the
course, so that from Week 2 onwards we build upon this foundation to
reach the three learning goals described in the next section!

\subsection{Main Textbooks / Resources}\label{main-textbooks-resources}

Unlike the case for topics like calculus or
\href{https://www.statlearning.com/}{statistical learning}, this field
is too new (and exciting! with new methods being developed
month-to-month) to have a single set of ``established'' textbooks. Thus,
the main collection of resources (books, papers, and explanatory videos)
we'll draw on for this class are available on the
\href{./resources.qmd}{resources page}. However, there are three
``core'' textbooks you can draw on which best align with the topics in
this course:

\begin{itemize}
\tightlist
\item
  Morgan and Winship, \emph{Counterfactuals and Causal Inference:
  Methods and Principles for Social Research} (Morgan and Winship 2015)
  {[}\href{https://www.dropbox.com/scl/fi/rmuw9yw0gflg9wprufzu4/Counterfactuals-and-Causal-Inference.pdf?rlkey=gpzpuqgmzgsdwo18j22pl6yiw&dl=1}{PDF}{]}

  \begin{itemize}
  \tightlist
  \item
    The book which comes closest to being an all-encompassing, single
    textbook for the class. It brings together different ``strands'' of
    causal modeling research (since each field---economics,
    bioinformatics, sociology, etc.---tends to use its own notation and
    vocabulary), unifying them into a single approach. The only reason
    we can't use it as \emph{the} main textbook is because it hasn't
    been updated since 2015, and most of the assignments in this class
    use computational tools from 2018 onwards!
  \end{itemize}
\item
  Angrist and Pischke, \emph{Mastering 'Metrics: The Path from Cause to
  Effect} (Angrist and Pischke 2014)
  {[}\href{https://www.dropbox.com/scl/fi/3xa9tfswmv1w8ez54prtk/Joshua-D.-Angrist-J-rn-Steffen-Pischke-Mastering-Metrics.pdf?rlkey=2w0ipphrgxrejd7ddkv1wd635&dl=1}{PDF}{]}

  \begin{itemize}
  \tightlist
  \item
    This book is included as the second of the three ``core'' texts
    mainly because, it uses the language of causality specific to
    Econometrics, the language that is most familiar to me from my PhD
    training in Political Economy. However, if you tend to learn better
    by example, it also does a good job of foregrounding specific
    examples (like evaluating charter schools and community policing
    policies), so that the methods emerging naturally from attempts to
    solve these puzzles when association methods like linear regression
    fail to capture their causal linkages.
  \end{itemize}
\item
  Pearl and Mackenzie, \emph{The Book of Why: The New Science of Cause
  and Effect} (Pearl and Mackenzie 2018)
  {[}\href{https://www.dropbox.com/scl/fi/0r7kg7hwibztvfnby8vdg/The-Book-of-Why.epub?rlkey=czdiw5mku5uamjdiwyyzlpmdr&dl=1}{EPUB}{]}

  \begin{itemize}
  \tightlist
  \item
    This book contrasts with the Angrist and Pischke book in using the
    language of causality formed within \emph{Computer Science} rather
    than Economics. It can be a good starting point especially if you're
    unfamiliar with the heavy use of \emph{diagrams} for scientific
    modeling---basically, whereas Angrist and Pischke's first instinct
    is to use (sometimes informal) \emph{equations} like \(y = mx + b\)
    to explain steps in the procedures, Pearl and Mackenzie's instinct
    would be to instead use something like
    \(\require{enclose}\enclose{circle}{\kern .01em ~x~\kern .01em} \overset{\small m, b}{\longrightarrow} \enclose{circle}{\kern.01em y~\kern .01em}\)
    to represent the same concept (in this case, a line with slope \(m\)
    and intercept \(b\)!).
  \end{itemize}
\end{itemize}

\subsection{Schedule}\label{schedule}

The following is a rough map of what we will work through together
throughout the semester; given that \textbf{everyone learns at a
different pace}, my aim is to leave us with a good amount of
\textbf{flexibility} in terms of how much time we spend on each topic:
if I find that it takes me longer than a week to convey a certain topic
in sufficient depth, for example, then I view it as a strength rather
than a weakness of the course that we can then rearrange the calendar
below by adding an extra week on that particular topic! Similarly, if it
seems like I am spending too much time on a topic, to the point that
students seem bored or impatient to move onto the next topic, we can
move a topic intended for the next week to the current week!

\begin{longtable}[]{@{}llll@{}}
\toprule\noalign{}
Unit & Week & Date & Topic \\
\midrule\noalign{}
\endhead
\bottomrule\noalign{}
\endlastfoot
\textbf{Unit 1}: The Language of Causal Modeling & 1 & May 21 &
\href{./w01/}{The Language of Causal Modeling} \\
& 2 & May 28 & \href{./w02/}{Probabilistic Graphical Models (PGMs)} \\
\textbf{Unit 2}: Matching Apples to Apples & 3 & Jun 4 &
\href{./w03/}{} \\
& 4 & Jun 11 & \href{./w04/}{} \\
& 5 & Jun 18 & \href{./w05/}{} \\
& 6 & Jun 25 & \href{./w06/}{} \\
\textbf{Midterm} & 7 & Jul 2 & \href{./w07/}{In-Class Midterm: Causal
Models and Statistical Matching} \\
\textbf{Unit 3:} Double-Debiased Machine Learning & 8 & Jul 9 &
\href{./w08/}{} \\
& 9 & Jul 16 & \href{./w09/}{} \\
& 10 & Jul 23 & \href{./w10/}{} \\
& 11 & Jul 30 & \href{./w11/}{} \\
\textbf{Unit 4}: Applications & 12 & Aug 6 & Final Project Details \\
& 13 & Aug 13 & \\
& & \emph{May 10 (Friday)} & \emph{{[}Deliverable{]}
\href{./final.qmd}{Final Project}} \\
\end{longtable}

\subsection{Assignments and Grading}\label{assignments-and-grading}

The main assignment in the course will be your \textbf{policy
whitepaper}, submitted at the end of the semester. However, there will
also be a midterm exam and a series of assignments which exist to let
you explore each of the modules of the course, in turn.

\begin{longtable}[]{@{}
  >{\raggedright\arraybackslash}p{(\linewidth - 4\tabcolsep) * \real{0.5500}}
  >{\raggedright\arraybackslash}p{(\linewidth - 4\tabcolsep) * \real{0.2500}}
  >{\raggedright\arraybackslash}p{(\linewidth - 4\tabcolsep) * \real{0.2000}}@{}}
\toprule\noalign{}
\begin{minipage}[b]{\linewidth}\raggedright
Assignment
\end{minipage} & \begin{minipage}[b]{\linewidth}\raggedright
Due Date
\end{minipage} & \begin{minipage}[b]{\linewidth}\raggedright
\% of Grade
\end{minipage} \\
\midrule\noalign{}
\endhead
\bottomrule\noalign{}
\endlastfoot
\href{https://classroom.google.com/c/NzQzMjkyMzg4NjIw/a/NzQ3NDA0MDQ5MjU3/details}{HW1:
Causal Modeling via DAGs and PGMs} & Friday, \textbf{May 30} & 10\% \\
\hyperref[]{HW2: TBD} & Friday, \textbf{February 21} & 10\% \\
Midterm & Wednesday, \textbf{February 28} & 30\% \\
\hyperref[]{HW3: TBD} & TBD & 10\% \\
\href{\%7B\%7B\%20var\%20hw4.url\%20\%7D\%7D}{HW4: TBD} & TBD & 10\% \\
\href{./final.qmd}{Final Project} & Friday, \textbf{May 10} & 30\% \\
\end{longtable}

\subsubsection{Homework Lateness Policy}\label{homework-lateness-policy}

After the due date, for each \textbf{homework} assignment, you will have
a grace period of 24 hours to submit the assignment without a lateness
penalty. After this 24-hour grace period, late penalties will be applied
based on the following scale (unless you obtain an excused lateness from
one of the instructional staff!):

\begin{itemize}
\tightlist
\item
  0 to 24 hours late: no penalty
\item
  24 to 30 hours late: 2.5\% penalty
\item
  30 to 42 hours late: 5\% penalty
\item
  42 to 54 hours late: 10\% penalty
\item
  54 to 66 hours late: 20\% penalty
\item
  More than 66 hours late: Assignment submissions no longer accepted
  (without instructor approval)
\end{itemize}

\phantomsection\label{refs}
\begin{CSLReferences}{1}{0}
\bibitem[\citeproctext]{ref-angrist_mastering_2014}
Angrist, Joshua D., and Jörn-Steffen Pischke. 2014. \emph{Mastering
'{Metrics}: {The Path} from {Cause} to {Effect}}. Princeton University
Press.

\bibitem[\citeproctext]{ref-morgan_counterfactuals_2015}
Morgan, Stephen L., and Christopher Winship. 2015. \emph{Counterfactuals
and {Causal Inference}: {Methods} and {Principles} for {Social
Research}}. Cambridge University Press.

\bibitem[\citeproctext]{ref-pearl_causality_2000}
Pearl, Judea. 2000. \emph{Causality: {Models}, {Reasoning}, and
{Inference}}. Cambridge University Press.

\bibitem[\citeproctext]{ref-pearl_book_2018}
Pearl, Judea, and Dana Mackenzie. 2018. \emph{The {Book} of {Why}: {The
New Science} of {Cause} and {Effect}}. Basic Books.

\end{CSLReferences}




\end{document}
